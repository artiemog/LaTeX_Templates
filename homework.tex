\documentclass{article}

%% Packages
% Lots of things below will break without this
\usepackage{amsmath}

% Needed for auto-size parens (below) and rcases (see example)
\usepackage{mathtools}

% Extra math symbols - full list at
% http://milde.users.sourceforge.net/LUCR/Math/mathpackages/amssymb-symbols.pdf
\usepackage{amssymb}

% Bold math characters (for matrices &c.)
% Use \bm in math mode to invoke, e.g. $\bm{A}$ or $\bm A$
\usepackage{bm}

% SI units (see package docs)
%\usepackage{siunitx}
%%

% Auto-size parens in math mode
% Use \br{} in place of ()
\DeclarePairedDelimiter\autobracket{(}{)}
\newcommand{\br}[1]{\autobracket*{#1}}

\title{ECE 3069 Spring 202x \\
       Homework 4}
\date{01 January 202x}
\author{Artemis Mog}

% ************************************************************************

\begin{document}
\maketitle

\section*{Problem 4}
\subsection*{Part (a)}

% Aligned multiline equation
\begin{equation*}
\begin{aligned}
\sigma &= q \mu_n n + q \mu_p p \\
&= q \mu_n n + q \mu_p \frac{{n_i}^2}{n} \\
\end{aligned}
\end{equation*}

% My "solution" for Scientific notation
\[ 3 \textsc{e} {-6} \]

% (Parenthetical) Matrices
% Other delimiters:
% bmatrix Bmatrix vmatrix Vmatrix matrix
% [ ]     { }     | |     || ||   (none)
\[
s \bm I - \bm A = \begin{pmatrix}
s & 3 \\
-3 & s+5 \end{pmatrix}
\]

% rcases example
\begin{equation*}
\begin{rcases}
\mu_n \\
\mu_p
\end{rcases}
\propto T^{-3/2}
\end{equation*}

% Referencing equations
% Document must be compiled (at least) twice or references will be broken
\begin{equation} \label{eq:sigma}
\sigma = q \cdot \br{n \alpha T^{-3/2} + p \beta T^{-3/2}}
\end{equation}
\begin{equation} \label{eq:n}
n = N_c e^{\frac{-\br{E_c - E_f}}{kT}}
\end{equation}
\begin{equation} \label{eq:Nc}
N_c = 2\br{\frac{m_n^*kT}{2\pi\hbar^2}}^{\frac{3}{2}}
\end{equation}
\begin{equation} \label{eq:p}
p = N_v e^{\frac{E_v - E_f}{kT}}
\end{equation}
\begin{equation} \label{eq:Nv}
N_v = 2\br{\frac{m_p^*kT}{2\pi\hbar^2}}^{\frac{3}{2}}
\end{equation}
Plugging in (\ref{eq:n}), (\ref{eq:Nc}), (\ref{eq:p}), and (\ref{eq:Nv}) into
(\ref{eq:sigma}), the resulting expression for $\sigma$ will vary with
$e^{-1/T}$.

% Boxing a final anwer
\[ \boxed{\mathcal{E} = -\frac{kt}{qx}} \]



\end{document}
